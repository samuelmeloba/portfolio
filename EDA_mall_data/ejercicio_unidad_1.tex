% Options for packages loaded elsewhere
\PassOptionsToPackage{unicode}{hyperref}
\PassOptionsToPackage{hyphens}{url}
%
\documentclass[
]{article}
\usepackage{amsmath,amssymb}
\usepackage{iftex}
\ifPDFTeX
  \usepackage[T1]{fontenc}
  \usepackage[utf8]{inputenc}
  \usepackage{textcomp} % provide euro and other symbols
\else % if luatex or xetex
  \usepackage{unicode-math} % this also loads fontspec
  \defaultfontfeatures{Scale=MatchLowercase}
  \defaultfontfeatures[\rmfamily]{Ligatures=TeX,Scale=1}
\fi
\usepackage{lmodern}
\ifPDFTeX\else
  % xetex/luatex font selection
\fi
% Use upquote if available, for straight quotes in verbatim environments
\IfFileExists{upquote.sty}{\usepackage{upquote}}{}
\IfFileExists{microtype.sty}{% use microtype if available
  \usepackage[]{microtype}
  \UseMicrotypeSet[protrusion]{basicmath} % disable protrusion for tt fonts
}{}
\makeatletter
\@ifundefined{KOMAClassName}{% if non-KOMA class
  \IfFileExists{parskip.sty}{%
    \usepackage{parskip}
  }{% else
    \setlength{\parindent}{0pt}
    \setlength{\parskip}{6pt plus 2pt minus 1pt}}
}{% if KOMA class
  \KOMAoptions{parskip=half}}
\makeatother
\usepackage{xcolor}
\usepackage[margin=1in]{geometry}
\usepackage{longtable,booktabs,array}
\usepackage{calc} % for calculating minipage widths
% Correct order of tables after \paragraph or \subparagraph
\usepackage{etoolbox}
\makeatletter
\patchcmd\longtable{\par}{\if@noskipsec\mbox{}\fi\par}{}{}
\makeatother
% Allow footnotes in longtable head/foot
\IfFileExists{footnotehyper.sty}{\usepackage{footnotehyper}}{\usepackage{footnote}}
\makesavenoteenv{longtable}
\usepackage{graphicx}
\makeatletter
\def\maxwidth{\ifdim\Gin@nat@width>\linewidth\linewidth\else\Gin@nat@width\fi}
\def\maxheight{\ifdim\Gin@nat@height>\textheight\textheight\else\Gin@nat@height\fi}
\makeatother
% Scale images if necessary, so that they will not overflow the page
% margins by default, and it is still possible to overwrite the defaults
% using explicit options in \includegraphics[width, height, ...]{}
\setkeys{Gin}{width=\maxwidth,height=\maxheight,keepaspectratio}
% Set default figure placement to htbp
\makeatletter
\def\fps@figure{htbp}
\makeatother
\setlength{\emergencystretch}{3em} % prevent overfull lines
\providecommand{\tightlist}{%
  \setlength{\itemsep}{0pt}\setlength{\parskip}{0pt}}
\setcounter{secnumdepth}{-\maxdimen} % remove section numbering
\usepackage{booktabs}
\usepackage{longtable}
\usepackage{array}
\usepackage{multirow}
\usepackage{wrapfig}
\usepackage{float}
\usepackage{colortbl}
\usepackage{pdflscape}
\usepackage{tabu}
\usepackage{threeparttable}
\usepackage{threeparttablex}
\usepackage[normalem]{ulem}
\usepackage{makecell}
\usepackage{xcolor}
\ifLuaTeX
  \usepackage{selnolig}  % disable illegal ligatures
\fi
\usepackage{bookmark}
\IfFileExists{xurl.sty}{\usepackage{xurl}}{} % add URL line breaks if available
\urlstyle{same}
\hypersetup{
  pdftitle={ejercicio\_unidad\_1},
  pdfauthor={Samuel Melo Balcázar},
  hidelinks,
  pdfcreator={LaTeX via pandoc}}

\title{ejercicio\_unidad\_1}
\author{Samuel Melo Balcázar}
\date{2025-01-24}

\begin{document}
\maketitle

\paragraph{Paso 1: Construcción de una tabla
descriptiva}\label{paso-1-construcciuxf3n-de-una-tabla-descriptiva}

Crear una tabla que contenga las siguientes columnas: Nombre de la
variable. Breve descripción de la variable. Clasificación según su
naturaleza. Clasificación según su origen.

\begin{table}
\centering
\begin{tabular}[t]{l|l|l|l}
\hline
nombres\_variables & Descripcion\_variables & clasificacion\_naturaleza & clasificacion\_origen\\
\hline
invoice\_no & Número de factura & Numérica & Transacción\\
\hline
customer\_id & id de cliente & Numérica & Cliente\\
\hline
gender & Sexo de los clientes & Categórica & Cliente\\
\hline
age & Edad & Numérica & Cliente\\
\hline
category & Artículo comprado & Categórica & Producto\\
\hline
quantity & Cantidad & Numérica & Transacción\\
\hline
price & Precio & Numérica & Transacción\\
\hline
payment\_method & Método de pago & Categórica & Transacción\\
\hline
\end{tabular}
\end{table}

\paragraph{Paso 2: Identificación y eliminación de
duplicados}\label{paso-2-identificaciuxf3n-y-eliminaciuxf3n-de-duplicados}

Identificar registros duplicados en la base de datos y eliminarlos.
Posteriormente, presentar un resumen con los resultados obtenidos por
variable.

\begin{longtable}[]{@{}
  >{\raggedright\arraybackslash}p{(\columnwidth - 18\tabcolsep) * \real{0.1038}}
  >{\raggedright\arraybackslash}p{(\columnwidth - 18\tabcolsep) * \real{0.1132}}
  >{\raggedright\arraybackslash}p{(\columnwidth - 18\tabcolsep) * \real{0.0660}}
  >{\raggedleft\arraybackslash}p{(\columnwidth - 18\tabcolsep) * \real{0.0377}}
  >{\raggedright\arraybackslash}p{(\columnwidth - 18\tabcolsep) * \real{0.0943}}
  >{\raggedleft\arraybackslash}p{(\columnwidth - 18\tabcolsep) * \real{0.0849}}
  >{\raggedleft\arraybackslash}p{(\columnwidth - 18\tabcolsep) * \real{0.0660}}
  >{\raggedright\arraybackslash}p{(\columnwidth - 18\tabcolsep) * \real{0.1415}}
  >{\raggedright\arraybackslash}p{(\columnwidth - 18\tabcolsep) * \real{0.1226}}
  >{\raggedright\arraybackslash}p{(\columnwidth - 18\tabcolsep) * \real{0.1698}}@{}}
\toprule\noalign{}
\begin{minipage}[b]{\linewidth}\raggedright
invoice\_no
\end{minipage} & \begin{minipage}[b]{\linewidth}\raggedright
customer\_id
\end{minipage} & \begin{minipage}[b]{\linewidth}\raggedright
gender
\end{minipage} & \begin{minipage}[b]{\linewidth}\raggedleft
age
\end{minipage} & \begin{minipage}[b]{\linewidth}\raggedright
category
\end{minipage} & \begin{minipage}[b]{\linewidth}\raggedleft
quantity
\end{minipage} & \begin{minipage}[b]{\linewidth}\raggedleft
price
\end{minipage} & \begin{minipage}[b]{\linewidth}\raggedright
payment\_method
\end{minipage} & \begin{minipage}[b]{\linewidth}\raggedright
invoice\_date
\end{minipage} & \begin{minipage}[b]{\linewidth}\raggedright
shopping\_mall
\end{minipage} \\
\midrule\noalign{}
\endhead
\bottomrule\noalign{}
\endlastfoot
I317333 & C111565 & Male & 21 & Shoes & 3 & 180051 & Debit Card &
2021-12-12 & Forum Istanbul \\
I294687 & C300786 & Male & 65 & Books & 2 & NA & Debit Card & 2021-01-16
& Metrocity \\
I337719 & C320928 & Male & 40 & Cosmetics & 5 & 2033 & Cash & 2021-01-18
& Istinye Park \\
I258204 & C167183 & Female & 54 & Clothing & 1 & 30008 & Credit Card &
2023-02-03 & Emaar Square Mall \\
I184085 & C175610 & Female & 33 & Cosmetics & 3 & 12198 & Credit Card &
2021-12-11 & Mall of Istanbul \\
\end{longtable}

\paragraph{Paso 3: Detección de valores atípicos o
inconsistencias}\label{paso-3-detecciuxf3n-de-valores-atuxedpicos-o-inconsistencias}

Identificar valores atípicos e inconsistencias para cada variable.
Generar una tabla con: El número de valores atípicos o inconsistentes
por variable. Los valores detectados como atípicos o inconsistentes.
Complementar esta información con gráficos adecuados y una tabla con
indicadores de tendencia central, posición y variabilidad.

\paragraph{Paso 4: Manejo de valores
atípicos}\label{paso-4-manejo-de-valores-atuxedpicos}

Tomar decisiones sobre cómo tratar los valores atípicos o
inconsistentes. Documentar las razones de las decisiones adoptadas y
presentar una tabla comparativa con los datos antes y después del
tratamiento. Incluir gráficos que ilustren el impacto de las decisiones
tomadas, junto con indicadores descriptivos actualizados.

\paragraph{Paso 5: Análisis de datos
faltantes}\label{paso-5-anuxe1lisis-de-datos-faltantes}

Calcular el porcentaje de datos faltantes por variable. Presentar los
resultados mediante al menos dos gráficos que destaquen las variables
afectadas y el porcentaje de datos faltantes correspondiente.

\paragraph{Paso 6: Test de hipótesis para datos
faltantes}\label{paso-6-test-de-hipuxf3tesis-para-datos-faltantes}

Implementar un test de hipótesis para determinar si el mecanismo de
generación de datos faltantes es MCAR (Missing Completely at Random).
Explicar los resultados obtenidos.

\paragraph{Paso 7: Imputación de datos
faltantes}\label{paso-7-imputaciuxf3n-de-datos-faltantes}

Aplicar técnicas de imputación para los datos faltantes, justificando
las decisiones adoptadas en cada caso. Complementar con gráficos que
ilustren el resultado de las imputaciones realizadas.

\paragraph{Paso 8: Análisis descriptivo
post-procesamiento}\label{paso-8-anuxe1lisis-descriptivo-post-procesamiento}

Realizar un análisis descriptivo de las variables después de la
limpieza, tratamiento de valores atípicos e imputación de datos
faltantes. Incluir:

Gráficos adecuados para cada variable. Indicadores estadísticos
relevantes. Evaluar posibles análisis bivariados que sean pertinentes en
el contexto del conjunto de datos.

\paragraph{Paso 9: Selección de resultados para el
informe}\label{paso-9-selecciuxf3n-de-resultados-para-el-informe}

Seleccionar gráficos e indicadores clave que resuman los hallazgos
obtenidos durante el análisis. Estos resultados serán utilizados en el
informe final.

\paragraph{Paso 10: Explicación de patrones y
tendencias}\label{paso-10-explicaciuxf3n-de-patrones-y-tendencias}

Elaborar un análisis final que explique los patrones y tendencias de
consumo en Estambul con base en los datos procesados.

\end{document}
